A Markov Decision Process (MDP) is a mathematical framework, developed in the 1950s, to optimise sequential decisions in a probabilistic model. Today, it is applied in a wide range of industries but seldom to manage fisheries despite its relevance to the governance of marine resources. In this paper, we applied a MDP to a delay-difference model calibrated to capture the dynamics of a trawl fishery targeting tiger prawn ({\it Penaeus esculentus}) in Moreton Bay, Australia. This bio-economic model of the fishery explored harvest control rules that maximized the economic profit of the fleet in response to variable economic and environmental conditions. Accounting for these uncertainties, a MDP suggests that the industry is operating at or its near optimal level given the regulatory and environmental constraints on the fishery. It shows that the Maximum Economic Yield (MEY) is achieved by adapting effort in response to abundance and economic conditions, notably by increasing fishing effort when the resource is more abundant. This MDP application to a tiger prawn fishery concludes that economic conditions and regulatory measures currently in place are constraining fishing effort well below the levels of effort that produce the Maximum Average Yield (MAY) or Maximum Sustainable Yield (MSY). Hence this stock is not currently at risk of becoming over-exploited. Simulations comparing the optimal strategy to other fishing strategies showed that (a) fishing the stock every year at MAY is un-profitable in the prevailing economic conditions, performing even worse than a random level of effort and (b) fishing every year at levels of effort 60\% lower (4,000--5,000 boat-days) than the maximum effort recommended by a MDP (7,000--8,000 boat-days) returns profits 89\% smaller than the optimal policy. By applying an MDP to this fishery, we contrast the notion of a static reference point (MSY and MEY) to the notion of dynamic changes in effort in response to stock abundance. The MDP policy yields greater profits than the other strategies due to its ability to adapt fishing effort in response to changing environmental and economic conditions.