Fishermen trawling in Moreton Bay catch a variety of species using 2 different types of gear: the beam-trawl is used up and down rivers to catch ... while otter-trawls are used in the Bay to catch at least 8 species of prawn. The fishery is further structured spatially with operators trawling specific areas and targeting species living in those areas: one can distinguish fishermen harvesting mostly greasybacks while other are catching mostly tiger prawns. Of all those species, tiger prawn is the most valuable. Therefore, it has been the object of most research effort to understand the causes of yield variations.

In recent years, this industry has come under increasing economic pressure: falling prawn prices over much of last decade has resulted in a substantial reduction in the number of boats operating in Moreton Bay \cite{Pascoe2017121}. This fall in price is largely attributable to increased supply of imported farmed prawns on the Australian market. Increasing costs of fishing, notably due to higher fuel prices (REF) and various legislations restricting fishing profitability such as prohibition to land certain by-catch species and access to specific areas previously open to fishing (REF) are likely to have contributed to the decline of the fishing industry in Moreton Bay.

Tiger prawn ({\it Penaeus esculentus}) is a short lived species \citep{garcia88c}. Mortality estimates for Moreton Bay suggests that few live for more than 1.5 years \citep{KienzleEtAl2015}. Every year, a new cohort recruits to the fishery between January and February \citep{KienzleEtAl2015} and provides the bulk of catch for that year. At the beginning of each fishing season, fishermen probe fishing grounds to detect if tiger prawns are present in sufficiently high size and abundance to determine begin targeting them. Similarly, the fishing season ends when fishermen switch to others species or areas in response to low densities resulting from stock depletion rending fishing un-economical. Analyses of fishermen's logbooks showed tiger prawn abundance increased in Moreton Bay over the last 20 years by a factor 2--3 \citep{Kienzle2014138}. A single-species stock assessment was developed to account for fishing and environmental effects on the dynamics of this species. Projections with the best parameterized stock assessment showed that the levels of fishing mortality are well within the biological potential of this stock: in reference to Maximum Sustainable Yield (MSY), this stock is under-exploited. The most recent results indicated that abundances of tiger prawns in Moreton Bay increased in the last 20 years in response to rising sea temperature induced by climate change: it is thought that the rise, of about 1 degree over the period, has created a more suitable environment for this tropical species to live in Moreton Bay which is located at the extreme South of its distribution on the East coast of Australia \citep{KienzleEtAl2017}.






