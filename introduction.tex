In fisheries, a harvest strategy is a rule to adjust fishing intensity in response to the abundance of an exploited natural resource to meet sustainability and economic or social objectives \citep{hil92b}. A harvest strategy is not a set of annual regulations; it has to be robust to unpredictable biological fluctuations affecting a stock. The productivity of short-lived species such as prawns is determined by the strength of recruitment each year and most effectively managed by effort control \citep{garcia88c}. There is a need to provide harvest strategies that balance the sustainability of fisheries with industry profitability. This problem is inherently stochastic because recruitment varies from year to year in response to environmental and fishing factors \citep{KienzleEtAl2017}. Deterministic approaches fail to account for the natural variability affecting such systems, leading to harvest rules inflexible to changes in recruitment. Markov Decision Processes (MDP) can address this problem elegantly because they provide a framework for decision-making in uncertain situations \citep{puterman_markov_1994}. This method accounts for the relationship between present and future decisions and provides an optimal state-dependent management strategy to the problem: an optimal policy. There have been a handful of MDP applications in fisheries research focusing on the number of fish left to spawn \citep{Mendelssohn1980a, LUDWIG1982273} which suggested that maintaining a constant escapement strategy was the optimal harvest strategy. These results have proved to be more of a hindrance than a help in designing good harvest strategies \citep{hil92b} because maintaining a constant stock-size is difficult to achieve in a naturally variable world and measuring stock abundance, before or after fishing, is challenging. The last decade has seen an explosion of successful usage of MDPs in artificial intelligence, providing a robust approach for autonomous systems to learn from and adapt to their environment by interacting with it \citep{Kochenderfer:2015:DMU:2815660}. Similarly, in the the field of species conservation there has been an increase in MDP usage, with applications in biological invasions \citep{firn_managing_2008,regan_optimal_2006}, disease management \citep{chades_general_2011}, release of biocontrol agents \citep{shea_optimal_2000}, harvesting \citep{williams_adaptive_1996}, migratory species \citep{nicol_adaptive_2013}, recovery of interacting species \citep{chades_setting_2012} and fire management regimes \citep{mccarthy_using_2001,possingham_optimal_1997,richards_optimal_1999}. \\

Moreton Bay is a large estuary-fed bay in Queensland situated near the southern limit of the distribution of tiger prawn ({\it Penaeus esculentus}) on the east coast of Australia (Fig.~\ref{fig:Map}). In this area, the stock of tiger prawns is managed by input control through gear restriction (fishing license, vessel size limited to up to 14m in length, net head-rope length) and temporal and spatial area closures to avoid conflicts with recreational users \citep{Pascoe2017121}. Currently, effort is capped in this fishery at a maximum of 32,100 boat-days, however not more than 4,900 days have been used in Moreton Bay since 2008. Contrary to other fisheries in Australia ({\it e.g.} \cite{Dichmont05012010}), the total amount of effort targeted at tiger prawn each year is not controlled by fisheries managers but by fishers who assess the abundance of new recruits on the fishing ground at the beginning of each fishing season to determine if and when it becomes economical to target tiger prawns (Moreton Bay Seafood Industry Association pers. comm.) effectively adapting their response to economic factors and the relative species abundance. The size of the fishing fleet has decreased by 60\% in the last 20 years as a result of increasing competition in the seafood market from other prawn fisheries, aquaculture production and increasing costs of fishing such as rising fuel prices \citep{Pascoe2017121, FAO2016}. Tiger prawn exploitation in Moreton Bay is assessed as being sustainable according to a single species delay-difference model which estimates all parameters to describe the dynamics of this fishery including recruitment \citep{KienzleEtAl2015, Kienzle2014138}. Maximum Sustainable Yield (MSY) or Maximum Average Yield (MAY\footnote{MSY and MAY are used interchangeably in this text. MAY was defined by \cite{Mace2000MSYa} as '... most fisheries scientists now interpret MSY in a more dynamic sense as the maximum average yield (MAY) obtained by applying a specific harvesting strategy to a fluctuating resource.'}) was estimated at 200 $\pm$ 50 tons per year, similar to that estimated by a multi-species model of this fishery \citep{NaWang2015a}. The largest uncertainty about the dynamics of this stock is recruitment, estimated with a delay-difference model, which fluctuates between years: the best predictor of recruitment estimated with a delay-difference model \citep{KienzleEtAl2015} found to date is water temperature which explains 69\% of its variability \citep{KienzleEtAl2017}. \\

The stock of tiger prawns in Moreton Bay does not have a harvest control rule (HCR). Although the exploitation of tiger prawn in Moreton Bay is sustainable at present, developing a harvest control rule may provide fisheries managers with insights into how and why fishers make decisions about when to fish, and aid managers to better understand and work with fishers to ensure that the stock remains sustainable without over-regulating the fishery. We applied a MDP to define a harvest control rule that maximizes the profit of the fishing industry while maintaining the capacity of the stock to withstand exploitation and regenerate itself year after year. Bio-economical models have been used previously to assess tiger prawn fisheries and the federal government of Australia has a history of using economic target reference points to manage some of its commercial fisheries \citep{Kompas2010a}. Previous analyses have ignored the effect of future recruitment variability on management advice because of the computational demands of bio-economic analyses (e.g. \cite{punt2010a}). The mathematical formulation of a MDP overcomes the computational problem and allows for the inclusion of the full spectrum of uncertainties. In particular, MDP overcomes the uncertainty about the magnitude of recruitment given the spawning stock size and can compute the optimal harvest control rule efficiently. Our MDP approach provides a Maximum Economic Yield (MEY) policy which defines optimal effort in response to the abundance of stock each year, accounting for the effect of fishing in the previous year, biological uncertainty in stock productivity as well as cost and revenue from fishing. We used this bio-economic model to investigate how different costs and revenue influence optimal levels of fishing effort, sustainability and economic profitability of the fishing fleet to provide a harvest control rule that achieves maximum economic yield while preserving the ability of the stock to regenerate itself. This analysis focused on a single species, the tiger prawn ({\it Penaeus esculentus}), because population dynamics models for other species caught by trawlers in this area have not been developed. Twenty-one years of catch and effort data recorded since 1990 have been used to identify the optimal harvest strategy presented in this document. This analysis was developed to account for the biological traits of the species, the characteristics of the fleet and environmental fluctuations affecting catch and recruitment \citep{KienzleEtAl2015}. It provides a fishing policy that maximizes surplus production and profitability of the fleet.\\

