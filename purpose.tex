% What is Markov Decision Processes - how it has been uses
A Markov Decision Process (MDP) is a method to determine the best course of actions to take in dealing with systems which reactions are unpredictable but quantifiable \citep{Puterman1994}. For a specific objective, an MDP provides a set of actions to be taken in specific situations (a policy) that achieves an optimal results over a period of time.

% Why did we decide to apply it to fisheries management in Moreton Bay
Fisheries management has long been recognised to fall within the class of problems that can be addressed by MDPs \citep{walters1986adaptive}. The intensity of fishing has a direct impact on the future abundance of the population of fish caught: too much fishing over long period of time has been shown repetitively to lead to exhaustion of these natural resources \citep{hilborn2011overfishing}. At a shorter timescale, the reaction of a stock to fishing is uncertain: stocks are often observed to produce good yields, even following large harvests, for reasons unrelated to fishing such as favourable environmental conditions. The survival of youngest age is largely un-predictable, induces large year to year variations in abundance of the new generation recruiting to the fishery. This is often the largest source of uncertainty fisheries manager have to deal with when deciding what actions to take. The problem is exhacerbated when catches are composed almost entirely of a single year-class, which abundance depends on recruitment, as is the case with short lived species such as prawns. Therefore the tiger prawn fishery in Moreton Bay seemed a good candidate to apply an MDP in order to understand what optimal harvest strategy this methods recommends.

%, the response of a fish stock to fishing pressure is uncertain mostly due to the very variable outcome of larval survival and recruitment 
