MDPs provide a well established and robust framework to compare various fishing strategies accounting for uncertainties in the system. The strength of this method is to embrace the dynamic nature of the fishery, and provide results that emphasize how adapting fishing effort to abundance maximizes profit. The range of effort maximizing profit was close to that estimated by a multispecies deterministic model reporting ${\rm E_{MEY}}$ ranging between 4.000 and 7.000 boat-days \citep{NaWang2015a}. The optimal harvest strategy is qualitatively and quantitatively in good agreement with changes in fishing effort observed in Moreton Bay over a 21-year period suggesting that fishers operate collectively at or close to the economic optimum. Larger observed annual fishing effort than suggested by the optimal harvest strategy at low level of recruitment might arise from differences in decision making between not so rational human behaviour and algorithms: the "sunken cost fallacy" as described in the field of behavioural economics has been widely observed in business ventures pursued beyond the point of economic rationality \citep{camerer2011advances}. The validity, realism and usefulness of the MDP results depend largely on the calibration of the delay-difference to logbook data as the MDP application itself is a deductive process based on the information encoded into the transitions matrix. \\

% devising an HCR based only on biological and ecological knowledge lead to policies that are inconsistent with the economic reality of the fishery, especially when the dynamics variation of abundance is not taken into account: caping effort at ${\rm E_{MAY}}$ leads to a target that is not reached a low levels of abundance because it is un-profitable and to loss of profits at high abundance. In other words, accounting for economics in designing policies provides a more adequate solutions to managing this and other fisheries.

% Good agreement between the data and the MDP results imply that the fishing fleet response to abundance and economics is very close to the optimal suggested by the MDP despite that fishing decisions were taken by each individual skipper and the model maximizes profit at the level of the fleet. 

The HCR that maximizes economic return to the industry while protecting the capacity of the tiger prawn stock to regenerate itself in Moreton Bay is to let effort, and catch, vary from year to year in response to unpredictable variations in stock size. Given recruitment of tiger prawns and stock size are the most uncertain aspects of the dynamics of this fishery, the best harvest strategy is neither to implement a total allowable catch \citep{Beddington1977} nor to impose season closures \citep{hil92b} but to allow fishers to adapt fishing effort to both biological and economical conditions. Managing prawn fisheries using effort instead of catch is made all the more important by their short-lived nature which makes abundance estimates very difficult \citep{parsons1993management}. The drawback of managing this fishery by input control is the resulting variability in catch potentially creating an over-supply of prawns and associated decline in prices. Nevertheless, the MDP policy presented here, which maximizes profit, has to be kept in perspective with the fact that only effort was allowed to vary in this study whereas more substantial increases in profit would have been achieved by also allowing catchability to increase. \\

%The economic model used is simple because the small size of the fishery precludes using complex (expensive) models.
% The economic model suggest increasing catch but does not take into account of the possible negative effect of larger catches on prawn prices.
The present implementation of the economic model, which is similar to that used in other bio-economic models ({\it e.g.} \cite{Dichmont05012010}), is simple in that it assumes fixed prices and costs over time. It is remarkable that despite its simplicity and the assumption that effort is adjusted by a single centralized entity maximizing its profit, the results are very similar to those observed from the sum of decisions made at the level of around 140 vessels over the period studied. This economic model was deliberately kept simple because the trawl fishery in Moreton Bay is too small to justify expensive data collection and analyses. As a consequence detailed economic information available to date is sporadic and imprecise preventing the development of more complex, dynamic, models. Had time series of prawn prices and fishing costs been collected over the last three decades, it would have been interesting to evaluate whether this simple economic model is capable of mimicking the declining fishing effort trajectory observed since 1990 perceived as resulting from the decline in prawn prices and increase in fuel costs over this period \citep{Pascoe2017121}. \\

This application of a MDP clearly demonstrates the influence of economical aspects of the fishery. Previous modelling work has evidenced that economic factors can increase fishing pressure on fish stocks in unregulated fisheries \citep{quin99b}. Instead, the present model of a regulated fishery suggests that economical factors and regulations act to limit the total effort expended in the fishery, preventing exploitation to become unsustainable. Further, effort levels required to achieve MAY appear unprofitable in the current economic and regulatory circumstances. This re-iterates the position long argued by some economists that a fishery maximizing its economic potential usually satisfies conservation objectives as well \citep{clark2017a}, at least in the low subsidy system characteristic of Australian fisheries. This position may be valid for Moreton Bay but not for the many fisheries in other jurisdictions that sustained perfectly economically viable businesses while increasing effort beyond ${\rm E_{MAY}}$ into a state of over-exploitation \citep{hilborn2011overfishing}.  \\

In practice, this work suggests that the tactics needed to implement the optimal harvest strategy each year have been performed successfully by the members of the fishing industry and the fishery management agency does not have to regulate effort in this fishery on a year-to-year basis as done for example in other prawn fisheries ({\it e.g.} \cite{Dichmont05012010}): it is not in fact the role of a harvest strategy \citep{hil92b}. Such tactical decisions are probably more efficiently made by fishers based on their knowledge of costs and landing prices specific to their own businesses. %rather than attempting to transfer this decision making process into an economic model of fisherman's behaviour that would be costly to develop, inflexible and difficult to maintain. Instead, this analysis suggests that the government agency responsible for managing this resource could play a more strategic role by developing policies that influence both side of the profit function in order to influence the exploitation of this resource. For example increasing revenue to attract new effort into this fishery, to develop the local economy and to increase profit from exploiting this resource by allowing more by-catch species to be commercialized. On the other side of the profit equation, tax incentives aimed at reducing fishing costs would have a similar effect and potentially benefit consumers too.\\

