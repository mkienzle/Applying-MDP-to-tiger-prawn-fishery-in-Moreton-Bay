State transition matrices which provide the probability that recruitment changes from low, medium or high level from year $t$ to year $t+1$ were calculated with the fishery model using 31 possible fishing intensities (Fig.~\ref{fig:BubblePlotOfTransitionMatrix}) ranging from 0 to 30,000 boat-days. The transition matrix when no fishing occurs (top-left in Fig.~\ref{fig:BubblePlotOfTransitionMatrix}) shows that recruitment at time $t$ has nearly equal probability ($0.31 \leq P(s_{t+1}|a_{1}, s_{t}) \leq 0.37$) to be followed by any magnitude of recruitment at time $t+1$ with a slightly larger chance (37\%) of a low level of recruitment being followed by a high level of recruitment than a low level of recruitment (31\%). At the other extreme of the spectrum of actions, where fishing effort used in the simulations is within the largest category considered in the simulations (29,000--30,000 boat-days), the recruitment is more likely to be low at time $t+1$ irrespective of recruitment at time $t$; the probability of a high recruitment following a low recruitment (24\%) is the smallest of all future states while the probability of a low recruitment being followed by a low recruitment is the largest (43\%). Intermediate levels of fishing effort, between 1,000 and 29,000 boat-days, resulted in transitions matrices intermediate to the two extreme cases described above. For example fishing with 19,000--20,000 boat-days, corresponding to effort level at MSY or MAY (E$_{\rm MAY}$): any level of recruitment in one year is equally probable to be followed by any level of recruitment a year after ($0.32 \leq P(s_{t+1}|a_{21}, s_{t}) \leq 0.35$). \\ %The choice of categories breaks to bin simulations results affected the results of the MDP because it influences these patterns in the transitions matrices. \\

The MDP showed that fishing strategies maximizing profit, for cases with cost of fishing $>$ \$0 per boat-day, were almost insensitive to the discount factor ($\gamma$): for $\gamma$ between 0.6 and 0.98, profit from exploiting tiger prawns in Moreton Bay is maximized by increasing effort as recruitment increases (Fig.~\ref{fig:SensitivityToDiscountFactor}). 
By contrast, the optimal policies obtained by maximizing catch, {\it i.e.} setting fishing costs to \$0 per boat-day, were more sensitive to the discount factor: at medium recruitment level, optimal efforts increased as the emphasis on catching fish in the present increased ($\gamma$ decreased). In the extreme case of valuing the present 1,000 times more than the future, $\gamma = 0.001$ (not shown in Fig.~\ref{fig:SensitivityToDiscountFactor}), the optimal strategy obtained was logically to fish at the highest level of effort possible for any level of recruitment. In the remainder of this section, the MDP solutions presented were computed for $\gamma = 0.8$ without loss of generality.\\

The method to calculate MAY estimated the maximum average catch over all possible levels of recruitment to equal to 217 $\pm$ 85 tonnes could be achieved with 19,500 boat-days of fishing effort per year (Fig.~\ref{fig:ProjectedCatchvsEffort}). This figure shows that $\rm{E_{MAY}}$ is extrapolated far removed from observed values. Increasing boat-days by a factor of four or more would have major impacts on the abundance of tiger prawns and other species caught by trawlers in Moreton Bay, in particular juvenile eastern king prawns. So rather than being proposed levels of fishing effort, these values are illustrative of what would be required to actually achieve MAY or MSY. Obviously, the economics of this fishery determine actual fishing effort, as follows.
%well beyond all levels of fishing effort observed between 1990 and 2010 in this fishery: it should be carefully considered for management purposes. Nevertheless, this estimate of $\rm{E_{MAY}}$ is consistent with the equivalent reference point calculated with the MDP: the MDP shows that maximum catch (Fig.~\ref{fig:SensitivityToDiscountFactor}, $\gamma=0.8$ and \$0 fishing costs) is achieved at low recruitment using 18,500 boat-days; at 19,500 boat-days at medium level of recruitment and at at 29,500 boat-days at high levels of recruitment. Contrary to the static nature of the calculations to estimate MAY, the MDP suggests that the best strategy to maximize tiger prawn catch is to apply more fishing effort in years of high recruitment rather than applying constant fishing effort ($\rm{E_{MAY}}$) irrespective of the fluctuations in recruitment. This dynamic interpretation of MSY indicated that such strategy yields 1 tonne more, none and 15 t more at low, medium and high levels of recruitment, respectively. \\

The MDP shows that including a profit function into the model with fishing costs of \$100 or more per boat-day is enough to constrain fishing effort below ${\rm E_{MAY}}$ (Fig.~\ref{fig:SensitivityToDiscountFactor}). The larger the costs, the smaller the fishing effort because profitable fishing is achievable only at higher abundance levels. The strategy that maximizes profit or Maximum Economic Yield (MEY) over an infinite number of years is to adapt effort in response to tiger prawn abundance and prevailing economic conditions. For example, the MDP shows that a low recruitment combined with high fishing costs (equal or above \$300 per boat-day) leads to no fishing on this species (Fig.~\ref{fig:SensitivityToDiscountFactor}, $\gamma=0.8$) due to negative profits when fishing occurs. 

The logbook data provide evidence that the fishing fleet, as a whole, has been behaving in agreement with the optimal strategy suggested by the MDP. First, the MDP suggested the industry should increase fishing effort when tiger prawn abundance is high to maximize profit: logbook data showed a statistically significant relationship (p-value=0.017) between effort and recruitment, explaining 23\% of the variability in effort (Fig.~\ref{fig:PlotEffortAgainstRecruitment})). Second, observed fishing effort and catches overlayed with the optimal MDP's policies (Fig.~\ref{fig:MDPonProfit-OverlayedWithObs} and ~\ref{fig:MDPonProfit-RecCatAgainstCatch-OverlayedWithObs}) suggested fishing effort deployed in Moreton Bay in response to tiger prawn abundance is consistent with optimal fishing strategies for costs of fishing varying between \$100 and \$400 per boat-day, with 80\% of observations falling between \$200 and \$300: the bio-economic models provide a more realistic description of the variation of effort in this fishery compared to the biological model that ignores economic factors and average catch across all recruitment variability to compute MAY. The average fishing effort catching tiger prawn between 1990 and 2010 is approximately 24\% of ${\rm E_{MAY}}$ (Fig.~\ref{fig:ProjectedCatchvsEffort}). Nevertheless, observed annual fishing effort appeared on average larger than predicted at low level of recruitment and smaller at high level: the observed variations of effort in response to abundance over time are not as steep as suggested by the MDP's optimal policies which assumed constant prawn prices and cost of fishing over the entire period. 

In comparison with traditional harvest strategies used in fisheries \citep{hil92b}, optimal strategies suggested by this MDP range from a constant escapement when cost of fishing is between \$300 and \$500 to proportional harvest when cost of fishing is \$0 or \$100 (Fig.~\ref{fig:MDPonProfit-RecCatAgainstCatch-OverlayedWithObs}). This suggests that economic factors do and can be used to alter fishing effort. Moreover, the economics of the fishery are constraining optimal fishing effort below $\rm{E_{MAY}}$ as illustrated by hypothetical scenarios varying prices of prawns from \$6 to \$40/kg (Fig.~\ref{fig:SensitivityToPrawnPrice}). Optimal fishing effort is positively correlated with profitability: a decline in prawn price by \$1 induces a decline in fishing effort by 300--500 boat-days per year; while increasing costs by \$1 per day of fishing reduces total effort by 20 boat-days a year. \\%A variation in prawn price by \$1/kg induces a variation in fishing effort of 500 boat-days per year.  \\

We used simulations to  compare the performance of the optimal strategy suggested by the MDP to alternative harvest strategies. Since this fishery does not have a HCR in place, we compared the outcome of applying an optimal strategy to that of applying (1) random effort; (2) constant low effort level (4,000--5,000 boat-days) and (3) constant effort at $\rm{E_{MAY}}$ (19,000--20,000 boat-days). These stochastic simulations, conducted over a 50 year period each, involved variable recruitment intensity and fishing effort varying or not according to the magnitude of recruitment (Fig.~\ref{fig:CompareVariousStrategies-1Simulation-TimeseriesOfStates} and ~\ref{fig:CompareVariousStrategies-1Simulation-TimeseriesOfActions}). The average yearly profit differed between strategies (Fig.~\ref{fig:CompareVariousStrategies-1Simulation-CumulativeReward}). A 1,000 replications of such simulations (Fig.~\ref{fig:CompareVariousStrategies-repeat1000times}) showed that constant effort at $\rm{E_{MAY}}$ and random effort strategies yielded negative profits in all simulations while the other two strategies were always profitable. The optimal strategy yielded the highest yearly profits, ranging from \$0.51 and \$0.87 million per year, with a median profit of \$0.70 million per year. The constant low effort strategy achieved profits ranging between \$0.45 and \$0.80 million per year, with a median profits of \$0.62 million (89\% of the optimal strategy). \\

% Comparison between optimal yield at constant (MSY) and variable fishing effort (MDP)

%Policies evaluations for a given price (12 \$/kg of prawn) and a range of costs of fishing, varying between 0 and 500 \$/boat-day, identified that the optimal fishing policy is to respond to a larger abundance by fishing more intensively (Fig.~\ref{fig:MDPonProfit}). \\
%The optimal strategy to maximize catch, given by maximizing the profit function fixing costs to zero, is to increase effort in response to larger abundance of the resource beyond fishing effort at MSY. This dynamic strategy achieves higher yields [ is is possible to calculate how much more catch than MSY ] and constrasts with the traditional, static,  view that Maximum Sustainable Yield is achieved at a fixed level of effort (show this on the graph).


%these discrepancies might arise from differences in optimal decision making between humans and algorithms induced by emotional responses such as the "sunken cost falacy".\\

%Varying fishing costs showed that when they are low, fishing can be profitable even at lower level of abundance.
%Varying fishing costs in the reward function showed that as they increase, there is a threshold abundance below which it becomes un-profitable to fish: the economics of fishing acts like a self protective mechanisms against biological over-fishing by stopping fishing at the lowest stock level. Varying both costs and prices of prawns showed that average fishing effort in the fishery declines as prawn prices decline and fishing cost increases (Fig.~\ref{fig:EffortSurface}). The total profit of the fishery is a declining function of cost of fishing at a given cost of fishing (Tab.~\ref{tab:AverageValueAsFctPricesAndCosts}) and an increasing function of price for a given cost. For example, at 12 \$/kg and 200 \$/boat-day, this stock will return an average profit of 5 M\$ per year. 


